\documentclass[margin]{res}

\setlength{\textwidth}{5.1in} % set width of text portion
\usepackage{hyperref}
\hypersetup{
    colorlinks=false,
    linkcolor=black,      
    urlcolor=black,
}

%\usepackage{setspace} % for \onehalfspacing and \singlespacing macros
%\onehalfspacing 

\usepackage{etoolbox}
\AtBeginEnvironment{quote}{}%\singlespacing\small}


\newcommand{\fullhrulefill}{%
	\vspace{.7\baselineskip}
	\hspace*{-\sectionwidth}\hrulefill%
}
\urlstyle{same}
\expandafter\def\expandafter\UrlBreaks\expandafter{\UrlBreaks%  save the current one
	\do\a\do\b\do\c\do\d\do\e\do\f\do\g\do\h\do\i\do\j%
	\do\k\do\l\do\m\do\n\do\o\do\p\do\q\do\r\do\s\do\t%
	\do\u\do\v\do\w\do\x\do\y\do\z\do\A\do\B\do\C\do\D%
	\do\E\do\F\do\G\do\H\do\I\do\J\do\K\do\L\do\M\do\N%
	\do\O\do\P\do\Q\do\R\do\S\do\T\do\U\do\V\do\W\do\X%
	\do\Y\do\Z}


\begin{document}

% Center the name over the entire width of resume:
 \moveleft.5\hoffset\centerline{\large\large\large\large\large\large\bf ALEXANDER SPANGHER}
 \vspace{\baselineskip}
 \moveleft.5\hoffset\centerline{ \large\large\large \textbf{\textit{Curriculum Vitae}}}
  \vspace{\baselineskip}
% Draw a horizontal line the whole width of resume:
 \moveleft\hoffset\vbox{\hrule width\resumewidth height 1pt}\smallskip
   \vspace{\baselineskip}
% address begins here
% Again, the address lines must be centered over entire width of resume:
% \moveleft.5\hoffset\centerline{} 
\moveleft.5\hoffset\centerline{237 McKendry Drive, Menlo Park, CA}
\moveleft.5\hoffset\centerline{Phone: (631) 487-7777} \moveleft.5\hoffset\centerline{Email: alexander.spangher@cmu.edu}
\moveleft.5\hoffset\centerline{Website: \textbf{https://alexander-spangher.com}}

\begin{resume}
\section{RESEARCH OBJECTIVES}
Language modeling in large-corpus domains to categorize and summarize texts, with a focus on journalistic practice and tooling.
Analytically studying modern media developments, including misinformation and online trolling. Methods: Bayesian modeling, deep learning, crowdsourcing and active learning.


\section{EDUCATION}
\fullhrulefill
\section{Doctoral Student}
{\bf Carnegie Mellon University}  \hfill {\sl 2018-present}\\
\-\hspace{.8cm} Graduate Studies, Electrical and Computer Engineering
\section{M.S.}
{\bf Columbia University}\footnote{Masters degrees pursued part-time while working full-time at the \textit{New York Times}.} \hfill {\sl2014-2018}\\
\-\hspace{.8cm} Master of Science in Data Science \\ 
\-\hspace{.8cm} Master of Science in Journalism
\section{B.S.}
% \-\hspace{.8cm} \textit{Note:} 
{\bf Columbia University} \hfill {\sl2010-2014} \\
\-\hspace{.8cm} Bachelor of Science in Neuroscience\\
\-\hspace{.8cm} Bachelor of Science in Computer Science

\section{HONORS}
\fullhrulefill
\section{Academic Honors}
\begin{itemize}
	\item[] \textbf{John Jay Scholar}. Columbia University. \hfill \textit{2010-2014}
	\item[] \textbf{Dean's List}. Columbia University. \hfill \textit{2011}
\end{itemize}

\section{Funding}
\begin{itemize}
	\item[] Columbia School of Journalism  Scholarship. \$78,000. \hfill \textit{2016-2017}
	\item[] \textit{New York Times} Tuition Scholarship \$32,000. \hfill \textit{2014-2018}
	\item[] John Jay Scholar Summer Funding. \$20,000 \hfill \textit{2011-2012}
	\item[] Intel STS Semi-finalist. \$10,000. \hfill \textit{2009}
%	\item[] National Merit Scholarship. \$1,000. \hfill \textit{2009}
\end{itemize}
%\end{itemize}

%\\ % corpus classification 
\section{GREs}
%\vspace{2\baselineskip}
\fullhrulefill
\section{}
\textbf{Verbal Reasoning:} \hfill 167/170\\
\textbf{Quantitative Reasoning:} \hfill 165/170\\
\textbf{Analytical Writing:} \hfill 4.5/6

%\vspace{2\baselineskip}
\section{PUBLICATIONS}
\fullhrulefill
\section{Academic\\Publications}
%\textbf{\\ Academic Publications}
\begin{enumerate}
	\item \textbf{Alexander Spangher}, Gireeja Ranade, Besamira Nushi, Adam Fourney, Eric Horvitz. Analysis of Strategy and Spread of Russia-sponsored Content in the US in 2017. \textit{International Conference for Web and Social Media, AAAI. \underline{Revise and Resubmit.}} 2018. \url{https://arxiv.org/pdf/1810.10033}
	%
	\item \textbf{Alexander Spangher}, Berk Ustun. Actionable Recourse in Linear Classification. \textit{Proceedings of the 5th Workshop on Fairness, Accountability and Transparency in Machine Learning, ICML. \underline{Accepted.}} 2018. \url{https://bit.ly/2FEj9pf}
	%
	\item \textbf{Alexander Spangher}, Berk Ustun. Actionable Recourse in Linear Classification in Practice. \textit{Workshop on Ethical, Social and Governance Issues in AI, 2018, NIPS. \underline{Accepted}} 2018. 
	%
	\item Berk Ustun, \textbf{Alexander Spangher}, Yang Liu. Actionable Recourse in Linear Classification. (Expanded Version). \textit{Conference on Fairness, Accountability and Transparency (FAT*), 2019, ACM. \underline{Accepted}} 2018. \url{https://arxiv.org/pdf/1809.06514.pdf}
	%
	\item Ryan L Boyd, \textbf{Alexander Spangher}, Adam Fourney, Besmira Nushi, Gireeja Ranade, James Pennebaker, Eric Horvitz. Characterizing the Internet Research Agency’s Social Media Operations During the 2016 US Presidential Election using Linguistic Analyses. \textit{Whitepaper, \underline{Published.}} \url{https://bit.ly/2SczIKt}
	%
\end{enumerate}

\section{In Preparation}
\begin{enumerate}
	\item \textbf{Alexander Spangher}, Gireeja Ranade, Adam Fourney, Besamira Nushi. Falling into the Rabbit Hole: Browsing Patterns Among Fake News Users. 2019.
	\item \textbf{Alexander Spangher}, Jia Zhang, Rahul Ramachandran, Manil Maskey, Patrick Gatlin, J.J. Miller, Sundar Christopher. Methodology for Building Scalable Knowledge Graphs using Pre-existing NASA Ontologies. 2019.
\end{enumerate}

\section{Newspaper Articles and Graphics (Selected)}
%\textbf{Selected Journalism Publications, Blogs and Interactive Graphics}
\begin{enumerate}
	%
	\item \textbf{Alexander Spangher}. Building the Next New York Times Recommendation Engine. \textit{The New York Times}. \url{https://nyti.ms/2zpGG5g}
	%
	\item \textbf{Alexander Spangher}. How Does This Article Make You Feel? Using data science to predict the emotional resonance of New York Times articles for better ad placement. \textit{The New York Times}. \url{https://nyti.ms/2PyHkcn}.
	%
	\item \textbf{Alexander Spangher}. What the Paris attacks tell us about how foreign news gets made. \textit{Columbia Journalism Review}. \url{https://bit.ly/2DIV2TH}
	\item \textbf{Alexander Spangher}. 19 Countries, 43 States, 327 Cities: Mapping The Times’s Election Coverage. \textit{The New York Times}. \url{https://nyti.ms/2ScnEbV}
	\item \textbf{Alexander Spangher}. Eye on the Prize: 100 years worth of Pulitzer Prize Winners by Race, Gender and Location. \textit{Columbia Journalism Review}. \url{https://bit.ly/2r2YEIT}
	\item \textbf{Alexander Spangher}. 3 Smart Data Journalism Techniques that can help you find stories faster. \textit{Medium}. \url{https://bit.ly/2DIUydH}.	
	\item For more articles and graphics, see: \url{alexander-spangher.com/data-vis.html}
\end{enumerate}

\vspace{5\baselineskip}
\section{PROFESSIONAL EXPERIENCE}
\vspace{\baselineskip}
\fullhrulefill
\section{Research Experience}
{\sl \bf Carnegie Mellon Univ.}, Mountain View, CA {\sl Ph.D. Student} \hfill {\it 2018-present}\\
\textit{Advisor:} Jia Zhang.\\
\textit{Knowledge Graph Construction for NASA Earth Sciences}.
\begin{itemize}
	\item Text-modeling using hierarchical topic modeling to improve and model existing NASA concept-ontologies.
	\item Text-matching and word-modeling using custom lexical parsing rules to extract datasets, variables and methods from papers.
	\item Visualization/tooling in D3.js with an emphasis on interpretability and ease of capturing user feedback.
\end{itemize}


{\sl \bf Microsoft Research}, Redmond, WA {\sl Research Intern} \hfill {\it Summer 2018}\\
\textit{Advisors:} Gireeja Ranade, Adam Fourney, Besamira Nushi, Eric Horvitz.\\
\textit{Large-scale analysis of user-behavior changes in response to misinformation}\\
\begin{itemize}
	\item Data analysis merging data from Facebook, Twitter and Microsoft. Causal modeling using counterfactual analysis.
	\item Text-modeling using TF-IDF to track search query changes over time.
	\item Intensive fact-checking, informal Congressional briefing, contact with staff of Congressman Adam Schiff (Representative, D-CA 28th District).
\end{itemize}


\section{Employment}
{\sl \bf New York Times}, NYC, NY {\sl Data Scientist} \hfill {\it 2014-2018}\\
\textit{Advisors:} Chris Wiggins, Jose Muanis Castro, Thompson Marzagao.\\
%\begin{itemize}
\textit{Collaborative Topic Models for Article Recommendations:} 
\begin{itemize}
	\item Created an improved article recommendation-engine by building a topic model to incorporate information from article-text and user clicks. Scale to millions of users and provide recommendations in real-time.
	\item Modeling: Custom-designed Bayesian model that extends Latent Dirichlet Allocation, coded in C++.
	\item Collaboration: Dr. David Blei, Jake Hoffman, and Prem Gopalan, Columbia University. See \url{https://arxiv.org/pdf/1311.1704.pdf.}
	\item Deployment: MySQL, WSGI API, Luigi data pipelines.
	\item Extensions: multi-armed bandit and contextual bandit algorithms.	
\end{itemize}

%\vspace{2\baselineskip}
\textit{Project Feels:} 
\begin{itemize}
	\item Modeled different emotions in \textit{New York Times} article body text. The purpose was to predict tragic, happy and polarizing articles for downstream decision-making.
	\item Modeling: 7 different deep learning architectures were tested alongside ensemble methods and other linear methods. 
	\item Data collection: Crowd-sourcing on Amazon Mechanical Turk, using active learning to select successive batches of articles to label.
	\item Deployment: Google Cloud Services (GKE, Datastore, BigQuery).
\end{itemize}

\textit{Newsroom Tools and Other:}
\begin{itemize}
	\item Used Latent Dirichlet Allocation and TF-IDF to build a related-articles feature for journalists doing research, directly into their publishing platform.
	\item Used simple character-level modeling and K-Means to cluster text-messages journalists received from Q\&A sessions with readers to facilitate responding.
	\item Used custom Bayesian model to perform newsletter recommendations for users.
	\item Used Random Forest and simple decision trees to create powerful and interpretable models of user retention likelihood.
\end{itemize}
 

%    See: \url{https://open.blogs.nytimes.com/2015/08/11/building- the-next-new-york-times-recommendation-engine/.}%\\
% Statistical analyses of audience segments, causality models for retention.
%\end{itemize}

\section{Open\\Source Code}
\textbf{Actionable Recourse Implementation} for \textbf{IBM Fairness 360 Project}. 
\begin{itemize}
	\item Implements Mixed-Integer Program (MIP) for providing actionable recourse auditing (see publication section above.) CPLEX and Pyomo based optimizer. 
	\item \textit{Contributors:} Berk Ustun, \textbf{Alexander Spangher}. \url{https://github.com/ustunb/actionable-recourse}
	\item \textit{Under development:} To be incorporated into IBM Fairness 360 project, an open-sourced project integrating different fairness and transparency algorithms.
\end{itemize} 

\textbf{Broca: A Battery of Natural Language Processing Methods} for \textbf{OpenSource News Fellows}
\begin{itemize}
	\item A pipeline system of organize a sequence of text-transformations. Automatic intermediate caching for time-saving and debugging.
	\item \textit{Contributors:} Francis Tseng, \textbf{Alexander Spangher}. \url{https://github.com/frnsys/broca}. 
	\item \textit{Blog:} Francis Tseng. Introducing Broca. \textit{OpenNews}. \url{https://bit.ly/2DVdrwH}.

\end{itemize}

\section{Languages and Frameworks} 
\begin{itemize}
	\item[] Python, SQL, D3.js, Javascript, Node.js, JQuery, Java, Scala, C, C++, C\#, CUDA, OpenCL. Google Cloud Services, Amazon Web Services. Spark, Databricks, Kubernetes, Drone, Docker. Sci-kit Learn, Tensorflow, Keras, PyOmo, CPLEX.
\end{itemize}

\vspace{2\baselineskip}
\section{PRESENTATIONS and CONFERENCES}
\vspace{2\baselineskip}
\fullhrulefill
\section{Speaking}
\begin{enumerate}
	\item \textbf{Alexander Spangher}. Project Feels and Actionable Recourse. Open Data Science Conference. San Francisco, California. 100+ attendees. October 2018.
	\item Adam Grant, \textbf{Alexander Spangher}. \textit{New York Times} Young Professionals Interview Series, New York City, NY. 100+ attendees. February 20th, 2018.
	\item Nicholas Kristof, \textbf{Alexander Spangher}, Hannah Cassius. \textit{New York Times} Young Professionals Interview Series, New York City, NY. 100+ attendees. March 18th, 2017.
	\item \textbf{Alexander Spangher}. Project Feels: Deep Text Models for Predicting the Emotional Resonance of \textit{New York Times} Articles. Open Data Science Conference. Boston, Massachusetts. 150+ attendees. April 30th-May 3rd, 2018.
	\item \textbf{Alexander Spangher}, Adam Kelleher. Recommender Systems in Digital Media. DataEngConf Meetup. New York City. 175+ attendees. March 15th, 2018.
	\item \textbf{Alexander Spangher}. Building the Next \textit{New York Times} Recommendation Engine. Data Engineering Conference. New York City, NY. 100+ attendees. December 13th-15th, 2015.
\end{enumerate}

\section{Adhoc Reviewer}
\begin{enumerate}
	\item Automated Knowledge Base Construction  2019 Conference. Amherst, Massachusetts, May 20th-21st 2019.
\end{enumerate}

\section{Teaching}
\begin{enumerate}
	\item \textit{Guest Lecture}. Mor Namaan, School of Information Science, Cornell Tech. May 20th, 2018.
	\item \textit{Guest Lecture}. Jonathan Stray, Graduate School of Journalism at Columbia University. February 20th, 2018.
	\item \textit{Guest Lecture}. Steven Coll, Graduate School of Journalism at Columbia University. December 4th, 2017. 
	\item \textit{Guest Lecture}. Jonathan Stray, Graduate School of Journalism at Columbia University. February 14th, 2017. 
	\item \textit{Guest Lecture}. Jonathan Stray, Graduate School of Journalism at Columbia University. February 2nd, 2015.
	\item \textit{Teaching Assistant}. Francis Champagne, The Developing Brain. Department of Psychology, Columbia University. Spring 2012.
\end{enumerate}

%\vspace{3\baselineskip}
\section{RELEVANT\\COURSEWORK}
\vspace{\baselineskip}
\fullhrulefill
\section{}
\textbf{Carnegie Mellon University, Relevant Courses, Doctoral Degree.}
\begin{enumerate}
	\item Osman Yagan. \textit{Applied Stochastic Processes.} ECE 18751. \hfill \textit{Spring 2019.}
	\item Jia Zhang.\textit{ Service Oriented Computing.} ECE 18655. \hfill \textit{Fall 2018}
\end{enumerate}
\textbf{Columbia University, Relevant Courses, Masters Degree.}
\begin{enumerate}
	\item Steven Coll. \textit{Investigative Techniques.} JOUR 6018. \hfill \textit{Fall 2017}
	\item John Paisley. \textit{Machine Learning for Data Science.} COMS 4721. \hfill \textit{Spring 2017}
	\item Adam Kelleher.\textit{ Causal Inference for Data Science.} COMS 4995. \hfill \textit{Spring 2017}
	\item Ronald Neath. \textit{Bayesian Statistics.} STAT 4640. \hfill \textit{Spring 2016}
	\item Eleni Drinea.\textit{ Algorithms for Data Science.} CSOR 4246. \hfill \textit{Fall 2016}
	\item Zoran Kostic. \textit{Comp. Signals \& Data Processing.} EECS 4750. \hfill \textit{Fall 2016}
	\item John Paisley.\textit{ Bayesian Models for Machine Learning.} EECS 6720. \hfill \textit{Fall 2016}
	\item Mark Hansen.\textit{ Data II.} JOUR 6015. \hfill \textit{Fall 2015} 
	\item James Fan. \textit{Deep Learning and Computer Vision.} EECS 6894. \hfill \textit{Spring 2015}
	\item Flavio Bartman. \textit{Linear Reg. and Time Series.} STAT 4440. \hfill \textit{Fall 2014}
	\item Alexandr Andoni. \textit{High Dimensional Data Analysis.} COMS 6998. \hfill \textit{Fall 2014}
\end{enumerate}
\textbf{Columbia University, Relevant Courses, Bachelor Degree.}
\begin{enumerate}
	\item Itshack Pe'er.\textit{ Machine Learning.} COMS 4771. \hfill \textit{Spring 2014}
	\item I-Han Hsiao. \textit{Data Visualization.} QMSS 4063. \hfill \textit{Spring 2014}
	\item Xi Chen. \textit{Analysis of Algorithms.} CSOR 4231. \hfill \textit{Spring 2014}
	\item Michael Collins.\textit{ Natural Language Processing.} COMS 4705. \hfill \textit{Fall 2013}
	\item Anargyros Papageorgiou. \textit{Comp. Linear Algebra.} COMS 3251. \hfill \textit{Fall 2013}
	\item Bodhisattva Sen. \textit{Probability and Statistical Inference}. STAT 4109. \hfill \textit{Fall 2013}
	\item Martha Kim.\textit{Fundamentals of Computer Systems}. CSEE 3827. \hfill \textit{Fall 2013}
	\item Salvatore Stolfo. \textit{Artificial Intelligence}. COMS 4701. \hfill \textit{Spring 2013}
	\item Seung Choi. \textit{Computer Science Theory.} COMS 3261. \hfill \textit{Spring 2013}
	\item Shlomo Hershkop. \textit{Data Structures in Java.} COMS 3134. \hfill \textit{Spring 2013}
	\item Jae Lee. \textit{Advanced Programming}. COMS 3157. \hfill \textit{Fall 2012}
	\item John Kender. \textit{Honors Intro to Computer Science.} COMS 1007. \hfill \textit{Fall 2012}
\end{enumerate}

\vspace{1\baselineskip}
\section{OTHER EXPERIENCES}
\vspace{1\baselineskip}
\fullhrulefill

\section{Diversity}
\begin{itemize}
	\item {\sl Co-chair and Founder:} NYT Young Professionals.
	\item {\sl Contributor:} Unpublished Black History \url{https://nyti.ms/2KvaGDM}.
\end{itemize}
\section{Professional Music Experiences}
\begin{itemize}
	\item {\sl Double Bassist} for critically-acclaimed off-Broadway play, \textit{The Dybbuk}. Review: \url{https://bit.ly/2FDSrNB}.
	\item {\sl Double Bassist}. New York Youth Symphony, Carnegie Hall. \hfill \textit{2006-2010}.
	\item \textit{Pianist} All-State Piano Recital and Orchestra. \hfill \textit{2008-2010}
\end{itemize}


\vspace{1\baselineskip}
\section{MEDIA MENTIONS}
\vspace{1\baselineskip}
\fullhrulefill
%\vspace{\baselineskip}
\section{(Selected)}
\begin{enumerate}
	\item \textit{BuzzFeed}. Peter Aldous. How Russia's Trolls Engaged American Voters. \url{https://bit.ly/2TFrhsB}
	\item \textit{WIRED}. Louise Matsakis. What Does a Fair Algorithm Actually Look Like? \url{https://bit.ly/2C8uyKN}.
	\item \textit{The Wall Street Journal}. Benjamin Mullin. New York Times Adapts Data Science Tools for Advertisers.  \textit{https://on.wsj.com/2sA4yof}.
	\item \textit{National Public Radio}. Le Show, February 18th, 2018. \url{https://bit.ly/2TGEpxF}. (A discussion on Project Feels.)
	\item \textit{Business Insider Japan}. Fumiaki Ishiguro. AI in Advertising at the \textit{New York Times} (Translated). \url{https://bit.ly/2Q57g0D}.
	\item \textit{Language Log, University of Pennsylvania}. Mark Liberman. Recommended for You. \url{https://bit.ly/2U7qktu}.
	\item \textit{KnightLab}. Shakeeb Asrar. A quick look at recommendation engines and how the New York Times makes recommendations. \url{https://bit.ly/2Sa6T15}.
	\url{https://bit.ly/2AIhkBQ}.
	\item \textit{Women Who Code}. Ema Kaminskaya. ODSC Event Reflections:
	\item[] \begin{quote}
			``Alex’s ability to captivate and connect with the audience was a sight to behold. The whole talk felt like an informal conversation between the presenter and 150+ people in the audience. That’s definitely a skill and a bit of a talent to manage such a big crowd in a very conversational way, encouraging questions and sparking curiosity.''% \\ --Ema Kaminskaya. ODSC Event Reflections. March 2018.  Published on \textit{Women Who Code}. \url{https://bit.ly/2AIhkBQ}
		\end{quote}
\end{enumerate}

\end{resume}
\end{document}




